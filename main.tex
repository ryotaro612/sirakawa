\documentclass[unicode, 14pt, aspectratio=169]{beamer}
\usetheme{isct}
\addbibresource{main.bib}
\date{\today}
\title{Infinispanとキャッッシュの分散}
\author{ryotaro612}
\newcommand\blfootnote[1]{%
  \begingroup
  \renewcommand\thefootnote{}\footnote{#1}%
  \addtocounter{footnote}{-1}%
  \endgroup
}

% \lstdefinestyle{gostyle}
% {
%   language = go,
%   keywordstyle = {\bfseries\color{brand}},
%   morekeywords = [1]{block},
%   keywords=[1]{type,block,struct,io.Reader,int64,fileReader,error}
% }

% \lstdefinestyle{pystyle}
% {
%   language = python,
%   keywordstyle = {\bfseries\color{brand}},
%   morekeywords = [1]{block},
%   keywords=[1]{type,block,struct,io.Reader,int64,fileReader,error}
% }
% % https://tex.stackexchange.com/questions/89574/language-option-supported-in-listings
% \lstdefinelanguage{typescript}{
%   keywords = {new, async, return}
% }
% \lstdefinestyle{tsstyle}
% {
%   language = typescript,
%   keywordstyle = {\bfseries\color{brand}},
% }
% \lstdefinelanguage{yaml}{
% }
% % JSON https://gist.github.com/ed-cooper/1927af4ccac39b083440d436d018d253
% \lstdefinelanguage{json}{
% }
\begin{document}

\begin{frame}[noframenumbering, plain]
  \titlepage
\end{frame}
% \section{デザインシステム ミツバチ}
% \begin{frame}
%   \frametitle{ミツバチ\supercite{mitsubachi-ui}}
%   {\large スピーダの次期デザインシステム}
%   \par
%   \begin{itemize}
%   \item Web Component\supercite{web-component}で実装
%   \item デザインの仕様はFigmaで管理
%   \item 実装されたコンポーネントはまだ4種類だけ
%   \item 採用中のプロジェクトはUBアカウントのみ
%   \end{itemize}  
% \end{frame}
% \begin{frame}
%   \frametitle{Web Component}
%   {\large 独自のHTMLタグを定義できる}
%   \par
%   \vfill  
%   {\large \textcolor{brand}{実装すること}}
%   \begin{itemize}
%   \item HTMLElementを継承するクラス、タグ名、属性値を宣言
%   \item DOMに接続したときのコールバック(子要素の追加など)
%   \item 属性値が代わるときのコールバック
%   \end{itemize}
% \end{frame}
% % \begin{frame}
% %   \frametitle{Web Componentの実装例}
% %   {\large HTMLElementの継承クラスを実装}
% % \end{frame}
% \section{Model Context Protocol (MCP)}
% \begin{frame}
%   \frametitle{Model Context Protcol (MCP)}
%   {\large LLMに情報やツールを提供するための仕様}
%   \begin{figure}[h]
%     \includegraphics[height=0.48\paperheight]{./img/mcp.png}
%     \caption[mcp]{MCPクライアントとMCPサーバの連携\footnote{図の引用元: \href{https://modelcontextprotocol.io/introduction}{modelcontextprotocol.io}}}
%     \label{fig:mcp}
%   \end{figure}
% \end{frame}
% \begin{frame}
%   \frametitle{MCPのTools}
%   {\large LLMに手続きを提供できる}
%   \vfill
%   \begin{center}
%     \textcolor{brand}{加算するtoolの実装例}
%     \lstinputlisting[basicstyle=\small, style=tsstyle]{./code/tool.ts}
%   \end{center}
% \end{frame}
% \begin{frame}
%   \frametitle{Figmaの非公式MCP\supercite{figma-mcp}のTool}
%   {\large FigmaのNodeのプロパティをYaml形式で返す}
%   \vfill
%   \begin{center}
%     ボタンのYaml(抜粋)と描画の比較
%   \end{center}
%   \begin{columns}
%     \begin{column}{0.6\linewidth}
%       \lstinputlisting[basicstyle=\scriptsize,xleftmargin=.4\linewidth]{./code/node.yml}
%     \end{column}
%     \begin{column}{0.4\linewidth}
%       \begin{figure}[h]
%         \includegraphics[width=1\linewidth]{./img/button.png}
%         \label{fig:button}
%       \end{figure}        
%     \end{column}
%   \end{columns}
%   Figmaのファイルは再帰的なNodeからできている
% \end{frame}
% \begin{frame}
%   \frametitle{MCPでやりたいこと}
%   {\large Figmaのデザインをミツバチで実装する}
%   \vfill
%   \begin{figure}[h]
%     \includegraphics[width=0.8\linewidth]{./img/overview.png}
%     \caption[mcp]{FigmaとミツバチのMCPの連携}
%     \label{fig:overview}
%   \end{figure}  
%   % \begin{itemize}
%   % \item FigmaのMCPが返すデザインをミツバチで実装
%   % \item ミツバチのコンポーネントを公開
%   % \item ミツバチのコンポーネントとFigmaのノードを対応づける
%   % \end{itemize}
% \end{frame}
% \begin{frame}
%   \frametitle{Custom Element Manifest\supercite{custom-element-manifest}}
%   {\large Web Componentの仕様を示したJSON Schema\supercite{json-schema}}
%   \vfill
%   \begin{itemize}
%   \item エディタ支援の強化、ドキュメントなどに使う
%     \begin{itemize}
%     \item StoryBookに情報を追加
%     \item API Viewer Element\supercite{api-viewer}による実装の可視化
%     \end{itemize}
%   \item \texttt{package.json}の\texttt{customElements}キーで配布する
%   \item Web Componentの実装からツール\supercite{analyzer}で生成できる
%   \item MCP以外にも上述の使い道がある
%   \end{itemize}
% \end{frame}
% \begin{frame}
%   \frametitle{Custom Element Manifestの例}
%   {\large ツールでコード内のコメントも取り出せる}
%   \lstinputlisting[basicstyle=\small]{./code/custom.json}  
% \end{frame}
% \begin{frame}
%   \frametitle{mitsubachi-ui-mcp\supercite{mitsubachi-ui-mcp}}
%   {\large 各コンポーネントのCustom Element Manifestを公開}
%   \par
%   \vfill
%   \begin{center}
%     \textcolor{brand}{実装}
%     \lstinputlisting[basicstyle=\scriptsize,style=tsstyle]{./code/ui-tool.ts}
%   \end{center}
% \end{frame}
% \begin{frame}[allowframebreaks,t]
%   \frametitle{参考資料}
%   \printbibliography
%   % \nocite{*}
% \end{frame}
\end{document}

